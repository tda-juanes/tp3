\section{Programación Lineal}

\subsection{Definiciones}

\subsubsection{Variables}

\begin{center}
    $Y_i$ := Variables binarias, indican si el elemento $A_{Y_i}$ forma parte
    de la solución. \\ (una por cada elemento en $A$)
\end{center}

\subsection{Modelo}

\subsubsection{Restricciones}

Necesitamos que haya por lo menos un elemento en cada subset, esto los
modelamos con una restricción por cada subset que toma la suma de las variables
asociadas a sus elementos y fuerza a esta a valer por lo menos uno:

\begin{equation}
    \sum_{Y \in B_i} Y \ge 1 \qquad \forall \quad B_i \in B
\end{equation}

\subsubsection{Funcional}

Estamos tratando de minimizar la cantidad de elementos en el resultado, por lo
que minimizamos el valor de la suma de las variables asociadas a los elementos
en el conjunto A.

\begin{equation}
    \min \{ \sum_{Y \in A} Y \}
\end{equation}

\subsection{Relajación}

Si dejamos que las variables $Y_i$ tomen valores reales el nuevo problema puede
ser resuelto en tiempo polinomial. Con esta solución podemos calcular una cota
inferior para el $k$ óptimo:

\begin{equation}
    \label{eq:k}
    k \ge \ceil{k_r} \qquad \text{con} \ k_r := \text{óptimo del problema
    relajado}
\end{equation}

Esto es porque al relajar las restricciones, la solución solo puede mejorar.
Además, si tomamos las variables cuyo valor excede $\frac{1}{b}$ obtenemos una
solución aproximada.

\subsubsection{Complejidad}

La complejidad del algoritmo con restricciones relajadas depende del algoritmo
utilizado por la librería \texttt{PuLP}. Si se utilizara el método simplex, si
bien es eficiente en la práctica tiene peor caso exponencial. Existen otros
algoritmos para resolver problemas de programación lineal que funcionan en
tiempo polinomial, como el algoritmo de Karmarkar.

\subsubsection{Calidad}

\[ \frac{A(I)}{z(I)} \le b \qquad \text{con} \ b := \max_{B_i \in B}(|B_i|) \]

\newpage

\begin{proof}

    Aprovechamos la ecuación \eqref{eq:k} para obtener una cota inferior para
    $z(I)$:

    \[ k_r \le z(I) \]
    \[ k_r b \le z(I) b \]

    También sabemos que nuestra aproximación $A(I)$, define que las variables
    con valor mayor o igual a $\frac{1}{b}$ serán 1. En el peor caso, todas las
    variables valen $\frac{1}{b}$ y pasan a valer 1, lo que nos deja con una
    función objetivo a lo sumo $b$ veces peor:

    \[ A(I) \le k_r b \, \implies \, A(I) \le z(I) b \]

    Si definimos $r(A)$ tal que $\frac{A(I)}{z(I)} \le r(A)$ obtenemos $r(A) =
    b$.
\end{proof}
