\section{Programaci\'on Lineal}

\subsection{Definiciones}

\subsubsection{Variables}

\begin{center}
    $Y_i$ := Variables binarias, indican si el elemento $A_{Y_i}$ forma parte
    de la soluci\'on. \\ (una por cada elemento en $A$)
\end{center}

\subsection{Modelo}

\subsubsection{Restricciones}

Necesitamos que haya por lo menos un elemento en cada subset, esto los
modelamos con una restricci\'on por cada subset que toma la suma de las
variables asociadas a sus elementos y fuerza a esta a valer por lo menos uno:

\begin{equation}
    \sum_{Y \in B_i} Y \ge 1 \qquad \forall \quad B_i \in B
\end{equation}

\subsubsection{Funcional}

Estamos tratando de minimizar la cantidad de elementos en el resultado, por lo
que minimizamos el valor de la suma de las variables asociadas a los elementos
en el conjunto A.

\begin{equation}
    \min \{ \sum_{Y \in A} Y \}
\end{equation}

