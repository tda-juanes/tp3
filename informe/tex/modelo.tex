\section{Programaci\'on Lineal}

\subsection{Definiciones}

\subsubsection{Variables}

\begin{center}
    $Y_i$ := Variables binarias, indican si el elemento $A_{Y_i}$ forma parte
    de la soluci\'on. \\ (una por cada elemento en $A$)
\end{center}

\subsection{Modelo}

\subsubsection{Restricciones}

Necesitamos que haya por lo menos un elemento en cada subset, esto los
modelamos con una restricci\'on por cada subset que toma la suma de las
variables asociadas a sus elementos y fuerza a esta a valer por lo menos uno:

\begin{equation}
    \sum_{Y \in B_i} Y \ge 1 \qquad \forall \quad B_i \in B
\end{equation}

\subsubsection{Funcional}

Estamos tratando de minimizar la cantidad de elementos en el resultado, por lo
que minimizamos el valor de la suma de las variables asociadas a los elementos
en el conjunto A.

\begin{equation}
    \min \{ \sum_{Y \in A} Y \}
\end{equation}

\subsection{Relajaci\'on}

Si dejamos que las variables $Y_i$ tomen valores reales el nuevo problema puede
ser resuelto en tiempo polinomial. Con esta soluci\'on podemos calcular una
cota inferior para el $k$ \'optimo:

\begin{equation}
    \label{eq:k}
    k \ge \ceil{k_r} \qquad \text{con} \ k_r := \text{\'optimo del problema
    relajado}
\end{equation}

Esto es porque al relajar las restricciones, la soluci\'on solo puede mejorar.
Adem\'as, si tomamos las variables cuyo valor excede $\frac{1}{b}$ obtenemos
una soluci\'on aproximada.

\subsubsection{Complejidad}

La complejidad del algoritmo con restricciones relajadas depende del algoritmo
utilizado por la librer\'ia \texttt{PuLP}. Si se utilizara el m\'etodo simplex,
si bien es eficiente en la pr\'actica tiene peor caso exponencial. Existen
otros algoritmos para resolver problemas de programaci\'on lineal que funcionan
en tiempo polinomial, como el algoritmo de Karmarkar.

\subsubsection{Calidad}

Aprovechando la ecuaci\'on \eqref{eq:k}, obtenemos una cota inferior para
$z(I)$:

\begin{equation*}
    k_r \le z(I)
\end{equation*}
\begin{equation*}
    k_r b \le z(I) b
\end{equation*}

Tambi\'en sabemos que nuestra aproximaci\'on $A(I)$, define que las variables
con valor mayor o igual a $\frac{1}{b}$ ser\'an 1. En el peor caso, todas las
variables valen $\frac{1}{b}$ y pasan a valer 1, lo que nos deja con una
funci\'on objetivo a lo sumo $b$ veces peor:

\begin{equation*}
    A(I) \le k_r b \, \implies \, A(I) \le z(I) b
\end{equation*}

Si definimos $r(A)$ tal que $\frac{A(I)}{z(I)} \le r(A)$ obtenemos $r(A) = b$.
