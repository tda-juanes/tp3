\documentclass{estilo}
\usepackage[spanish]{babel}
\usepackage{graphicx}
\usepackage{float}
\usepackage{amsmath,amsthm} % para los vectores columnas
\usepackage{amsfonts}       % para las negrita de pizarra
\usepackage{amssymb}        % para simbolos matematicos
\usepackage{hyperref}       % para utilizar referencias
\usepackage{multirow}       % para las tablas
\usepackage{dsfont}
\usepackage{listings}
\usepackage{xcolor}
\definecolor{codegreen}{rgb}{0,0.6,0}
\definecolor{codegray}{rgb}{0.5,0.5,0.5}
\definecolor{codepurple}{rgb}{0.58,0,0.82}
\definecolor{backcolour}{rgb}{0.95,0.95,0.92}
\lstdefinestyle{mystyle}{
    backgroundcolor=\color{backcolour},
    commentstyle=\color{codegreen},
    keywordstyle=\color{magenta},
    numberstyle=\tiny\color{codegray},
    stringstyle=\color{codepurple},
    basicstyle=\ttfamily\footnotesize,
    breakatwhitespace=false,
    breaklines=true,
    captionpos=b,
    keepspaces=true,
    numbers=left,
    numbersep=5pt,
    showspaces=false,
    showstringspaces=false,
    showtabs=false,
    tabsize=2
}
\lstset{style=mystyle}

\usepackage{enumitem,multicol,setspace}
\newcounter{subenum}[enumi] % para las multicolumnas
\renewcommand{\thesubenum}{\arabic{subenum}}
\usepackage[nomessages]{fp}
\FPeval\thecolwidth{round(1/4:4)}% Specify number of columns -> column width
\newcommand{\newitem}[1]{%
  \refstepcounter{subenum}%
  \parbox{\dimexpr\thecolwidth\linewidth-.5\columnsep}{%
    \makebox[\labelwidth][r]{(\thesubenum)\hspace*{\labelsep}}%
    #1}\hfill%
}

\usepackage{scalerel,stackengine} % para el sombrero
\stackMath
\newcommand\rhat[1]{%
\savestack{\tmpbox}{\stretchto{%
  \scaleto{%
    \scalerel*[\widthof{\ensuremath{#1}}]{\kern-.6pt\bigwedge\kern-.6pt}%
    {\rule[-\textheight/2]{1ex}{\textheight}}%WIDTH-LIMITED BIG WEDGE
  }{\textheight}% 
}{0.5ex}}%
\stackon[1pt]{#1}{\tmpbox}%
}
\parskip 1ex

\usepackage{mathtools}      % floor y ceil
\DeclarePairedDelimiter\ceil{\lceil}{\rceil}
\DeclarePairedDelimiter\floor{\lfloor}{\rfloor}

\usepackage[style=authoryear-comp]{biblatex}


\begin{document}
\maketitle

\justifying{}

\newpage
\section{Introducción}

En el presente trabajo práctico se demuestra que el \textit{Hitting-Set
Problem} como problema de decisión es NP-Completo, se desarrolla una solución
con un algoritmo de \textit{Backtracking} para el mismo, como problema de
optimización, y se analizan posibles soluciones aproximadas.

\section{Definición del \textit{Hitting-Set Problem}}

Dado un conjunto de elemento $A$ de $n$ elementos, $m$ subconjuntos $B_1, B_2,
\ldots, B_m$ de $A \ (B_i \subseteq A \\ \forall \ i)$, queremos el subconjunto
$C \subseteq A$ de menor tamaño tal que $C$ tenga al menos un elemento de cada
$B_i$ (es decir, $C \cap B_i \ne \emptyset$).

\subsection{Como problema de decisión}

Dado un conjunto de elemento $A$ de $n$ elementos, $m$ subconjuntos $B_1, B_2,
\ldots, B_m$ de $A \ (B_i \subseteq A \\ \forall \ i)$, y un número $k$, ¿existe un
subconjunto $C \subseteq A$ con $|C| \le k$ tal que $C$ tenga al menos un
elemento de cada $B_i$ (es decir, $C \cap B_i \ne \emptyset$)?

\section{Demostraciones}

Un problema de decisión $P$ es \textit{NP}-Completo si $P$ pertenece a
\textit{NP} y $P$ es \textit{NP}-Difícil.

\subsection{\textit{Hitting-Set Problem} está en \textit{NP}}

Un problema pertence a $NP$ si una solución al mismo puede ser verificada en
tiempo polinomial por una máquina de Turing determinística, o alternativamente,
el problema puede ser resuelto en tiempo polinomial por una máquina de Turing
no determinística.

El siguiente algoritmo es un posible verificador de soluciones del
\textit{Hitting-Set Problem}:

\lstinputlisting[language=Python]{code/verify.py}

El algoritmo es de tiempo polinomial porque $|B| = m$, $B_i \subseteq A \
\forall \ i$ y $C \subseteq A$, por lo que $|B_i| \le |A|$ y $|C| \le |A|$ y la
complejidad es $\mathcal{O}(N^2m)$ con $N = |A|$. Esto es porque la complejidad
del verificador esta dada por la mayor de las siguientes:

\begin{itemize}
    \item una comparación del tamaño de $C$ contra $k$ ($\mathcal{O}(1)$).

    \item un loop por todos los elementos de $C$ verificando que esten en $A$
    ($\mathcal{O}(N^2)$).

    \item un loop por $m$ elementos (la cantidad de subsets), con otro loop
    adentro de a lo sumo $N$ elementos cada uno (la cantidad de elementos en
    $B_i$) que chequea si algún elemento esta en $C$ (de tamaño a lo sumo $N$),
    que si $C$ es una lista es una operación de tiempo lineal
    ($\mathcal{O}(N^2m)$).
\end{itemize}

\subsection{\textit{Hitting-Set Problem} es \textit{NP}-Difícil}

Para demostrar que el problema es \textit{NP}-Difícil realizamos una reducción
polinomial de un problema \textit{NP}-Completo a nuestro problema,
\href{https://en.wikipedia.org/wiki/Vertex_cover}{\underline{Vertex Cover}}.

Un \textit{Vertex Cover} $V'$ de un grafo no dirigido $G = (V, E)$, es un
conjunto de vertices $V' \subseteq V$, tal que para toda arista $(u, v) \in E$,
$u \in V'\ \lor\ v \in V'$, o lo que es lo mismo, todas las aristas del grafo
$G$ tienen por lo menos una esquina en $V'$. La versión de decisión del
problema se trata de determinar si existe un \textit{Vertex Cover} de a lo sumo
$k$ vértices.

Para reducir este problema al \textit{Hitting-Set Problem} creamos un subset
$B_i = \{ u, v \}$ por cada arista $(u, v) \in E$, $A = V$ y $k = k$. Luego
tomamos la solución del \textit{Hitting-Set Problem} $C$ y con ella generamos
la solución del \textit{Vertex Cover} $V' = C$:

\lstinputlisting[language=Python]{code/vertex_cover.py}

Esta reducción se puede realizar en $\mathcal{O}(V^2)$, que es el costo de
crear un subset por cada arista en el grafo $G$.

\begin{figure}[h]
    \centering
    \begin{subfigure}{0.4\textwidth}
        \centering
        \includegraphics[width=0.585\linewidth]{img/vertex-cover.png}
        \caption{Input de \textit{Vertex Cover}}
    \end{subfigure}
    \begin{subfigure}{0.4\textwidth}
        \centering
        \includegraphics[width=0.6\linewidth]{img/hitting-set.png}
        \caption{Input de \textit{Hitting-Set}}
    \end{subfigure}
    \caption{Reducción de \textit{Vertex Cover} a \textit{Hitting-Set Problem}}
\end{figure}

En el caso de la figura, al generar un set con los dos vértices que componen a
cada una de las aristas, el problema de \textit{Vertex Cover} para el grafo dado
se vería reducido al \textit{Hitting-Set Problem} para los sets $\{A,B\}$,
$\{A,C\}$ y $\{C,D\}$.

Se puede comprobar que el conjunto de soluciones para el \textit{Hitting-Set Problem}
es $\{\{A,D\},\{A,C\},\{C,B\}\}$, que no es ni más ni menos que el conjunto de
soluciones para \textit{Vertex Cover}.
\newpage
\section{Programaci\'on Lineal}

\subsection{Definiciones}

\subsubsection{Variables}

\begin{center}
    $Y_i$ := Variables binarias, indican si el elemento $A_{Y_i}$ forma parte
    de la soluci\'on. \\ (una por cada elemento en $A$)
\end{center}

\subsection{Modelo}

\subsubsection{Restricciones}

\begin{equation}
    \sum_{Y \in B_i} Y \ge 1 \qquad \forall \quad B_i \in B
\end{equation}

\subsubsection{Funcional}

\begin{equation}
    \min \{ \sum_{Y \in A} Y \}
\end{equation}


\section{Aproximaci\'on}

Para encontrar una soluci\'on aproximada al \textit{Hitting-Set Problem},
proponemos el siguiente algoritmo \textit{Greedy}:

\lstinputlisting[language=Python]{code/greedy.py}

En este algoritmo se itera por todos los elementos de los subsets para calcular
las frecuencias, si los subsets tienen a lo sumo $N$ elementos cada uno, esto
tiene complejidad $\mathcal{O}(Nm)$, siendo $m$ la cantidad de subsets. Luego
se itera quitando por lo menos un subset de la lista de subsets cada vez, es
decir $\mathcal{O}(m)$ iteraciones, y en ese \textit{loop} se itera por todos
los subsets buscando aquellos que contengan al de mayor frecuencia realizando
operaciones de costo lineal en la cantidad de elementos del subset, como lo son
buscar un elemento, iterar por el mismo y remover un elemento en cualquier
posici\'on, lo que nos deja con una complejidad de $\mathcal{O}(m^2N)$.

\newpage
\section{Mediciones}

Se realizaron mediciones en base a crear sets de distinta cantidad de
elementos, con la cantidad de subsets y la cantidad de elementos por subset
siendo proporcional a la cantidad de elementos.

Para asegurar la validez de las comparaciones todas las mediciones fueron
realizadas sobre los mismos sets de datos para los métodos que están siendo
comparados. Los elementos fueron generados por los valores pseudoaleatorios del
lenguaje (el módulo \texttt{random}) con la misma \textit{seed}.

\subsection{Backtracking vs. Programación Lineal}

\begin{figure}[H]
    \centering
    \includegraphics[width=1\textwidth]{img/backvslp.png}
\end{figure}

Backtracking obtiene mejores tiempo de ejecución que programación lineal para
encontrar la solución óptima.

\subsection{Algoritmos de aproximación}

\begin{figure}[H]
    \centering
    \includegraphics[width=0.49\textwidth]{img/greedy.png}
    \includegraphics[width=0.49\textwidth]{img/pl_rlx.png}
\end{figure}

Notar la diferencia de tiempos (eje y) entre greedy y programación lineal.

\subsection{Cota de la relajación lineal}

Se realizaron mediciones para verificar la cota calculada empiricamente con un
valor de $r(A) = 20$:

\begin{figure}[H]
    \centering
    \includegraphics[width=0.8\textwidth]{img/cota.png}
\end{figure}

Vemos que $\frac{A(I)}{z(I)}$ se mantiene por debajo de la cota calculada. Cabe
aclarar que a partir de los 175 elementos se utilizó una cota inferior para
$z(I)$, por lo que se ve un pico en el gráfico a partir de ahí, esta cota
inferior fue calculada por el mismo algoritmo de programación lineal relajada
porque los volumenes de datos eran inmanejables para el algoritmo exacto.

\newpage
\section{Conclusiones}

Aca ir\'ian sus conclusiones.



\newpage
\end{document}
